%%Énoncé
Quelle est la complexité temporelle de votre algorithme MaxBandWidth(G, a, b)
(décrit à la question 5) ? Précisez notamment les hypothèses éventuelles sur l’implémentation
des structures de données utilisées, dans la mesure où ces hypothèses
seraient importantes pour justifier la complexité annoncée.%%Auteur
(Alexandre)\\
%%Réponse

Soit n et m, respectivement, le nombre de nœuds et d'arêtes du graphe G. Nous représentons le graphe G avec une "adjacency list structure". Son implémentation n'influence pas à elle seule le complexité de l'algorithme de la fonction $MaxBandWidth(G, a, b)$, il faut également tenir compte de la file de priorité où les mises à jour sont réalisées en log(n). Donc, nous pouvons déterminer la complexité de l'algorithme de la fonction $MaxBandWidth(G, a, b)$ comme étant égale à O((n+m)log(n)).
